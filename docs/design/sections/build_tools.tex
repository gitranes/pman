\section{Project structure and build tools}

\subsection{CMake build system}

CMake was an rather obvious choice for a cross-platform build system, as the
only real other candidate would have been Make. Modern CMake is well integrated
with different C compiler options and tools.

\subsubsection{Folder structure}

The \pman project is split to its architectural components using concise
folder structure. Each subfolder/library will contain their own \textit{CMakeLists.txt}
that specifies how the possible source files in the folder are built. The folders
of the project with their purposes are listed in table \ref{tab:folder}.

\begin{table}[h]
    \centering
    \begin{tabular}{|c|c|}
        \hline
        Folder & Purpose \\

        \hline
        \textit{cmake/} & CMake scripts and modules \\

        \hline
        \textit{include/} & Header files (\textit{.h}) \\

        \hline
        \textit{src/} & Source files (\textit{.c}) \\

        \hline
        \textit{tests/} & Unit test files \\

        \hline
    \end{tabular}
    \caption{Folder structure of \pman}
    \label{tab:folder}
\end{table}

\subsubsection{Clang and C17}

The C compiler for the project was chosen to be Clang due to its superior tool
capabilities and integration that is heavily utilized. Clang provides many
highly capable static analysis and linting tools, which for a large C project
are more than welcome.

The newest complete C standard, C17, was chosen due to all of the useful
features provided in C11. \pman does not have portability constraints,
so choosing the newest C standard poses no clear downsides.

\subsubsection{CTest and unit testing}

CTest is a unit test driver integrated with CMake, so it was a self-explanatory
choice for the project. CTest implementation for the project can be seen in
\textit{tests/} subfolder of the project.

The \pman project utilizes unit testing heavily, which is why a Unity
library is used to extend the capabilities of the unit tests. Unity provides
easy-to-use unit testing interface similar to Google Test etc. while also fully
supporting C.

\subsection{Static analysis and linting}

As security is one of the key design principles of \pman, the build
system integrates several tools to facilitate this principle. Static analysis
and linting are used during the building process to eliminate as many bugs and
potential security risks as possible, as early as possible.

\subsubsection{Compiler warnings}

All C compilers have a way of generating warnings from a possibly buggy code,
and Clang is no exception. To utilize warnings to their full effect, the
\pman build process enables every single feasible warning provided by
the Clang compiler. Additionally, every warning produced by the compiler is
transformed in to an error, which forces fixing each and every one of them.

\subsubsection{Clang-Tidy linter and static analyzer}

\texttt{clang-tidy} is a linting tool for C/\CC, which provides an extensible
framework for diagnosing typical programming errors. These include style
violations, interface misuse, and bugs.

\texttt{clang-tidy} is integrated in to the build process of \pman,
and all of its warnings are interpreted as errors. Additionally, the project
also supports using CLion IDE, which further integrates with \texttt{clang-tidy}.

The utilized checks and project styling parameters are listed in the
\textit{.clang-tidy} file at the root of the project. Along with enabling
almost all possible checks, a consistent naming scheme is enforced.

Along with the \texttt{clang-tidy} mentioned in the previous section, Clang
also provides a powerful static analyzer purpose-built for C/\CC code. For such
a small project as \pman, the only minor drawback is increase in
compilation-time, which is well worth the benefits.

The Clang Static Analyzer integrates with \texttt{clang-tidy}, and therefore,
is also ran during the normal build process. The project also implements a
script for running the static analyzer separately in \textit{scripts/} folder.

