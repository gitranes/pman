\section{Introduction}

This documents serves as an introduction to the architecture and design of
\pman, which is a command-line based password manager for Linux
platforms. \pman will be written in C and it revolves around managing a
file-based password database that the user can manage with different commands.

Before the introduction of the architecture, the build system of \pman
is shortly described. This will include the chosen build system, build tools,
some relevant compiler options, and the different linters and static analyzers
used in the project.

Given the relatively simplicity \pman as a program, the architecture
is succinct and contains only a few critical architectural views. Despite this,
making an architecture design is critical, as then secure design and, most
importantly, secure handling of passwords can be emphasized in the construction
of \pman as early as possible.

The utilized architecture views consist of activity, logical and deployment views.
The activity view was decided instead of the usual use-case view due to only real
user being the user of the command-line. Logical view was an obvious decision given
the security requirements described in the previous paragraph. Finally, the
deployment view provides a look in to the different libraries that will be
built as a part of the \pman project.

At the very end, this document will also go in to the detailed design of
\pman, which will include detailed interfaces of the different components
described in the architecture description including the component's respective
interface documentations. Even though C does not support classes, to which UML
is quite heavily biased to, class diagrams will be utilized in the interface
descriptions.
